\documentclass[11pt]{article}
\usepackage{graphicx}
%\usepackage{grffile}
%\usepackage{longtable}
%\usepackage{wrapfig}
%\usepackage{rotating}
%\usepackage[normalem]{ulem}
\usepackage{amsmath}
\usepackage{textcomp}
\usepackage{amssymb}
%\usepackage{capt-of}
\usepackage{hyperref}
%\usepackage{fontspec} -- was added by org-mode
%\usepackage{newfloat}
%\usepackage{minted} -- was added by org-mode
\usepackage{booktabs}
\newcommand{\gr}[1]{\mathfrak{#1}}
\newcommand{\GG}{\gr{G}}
\newcommand{\VV}{\mathbb{V}}
\newcommand{\R}{\mathbb{R}}
\newcommand{\void}{\varnothing}
\newcommand{\unit}{\bullet}
\newcommand{\one}{\mathbf{1}}
\newcommand{\two}{\mathbf{2}}
\newcommand{\three}{\mathbf{3}}
\DeclareMathOperator{\reduce}{reduce}
\DeclareMathOperator{\map}{map}
\DeclareMathOperator{\zipWith}{zipWith}
\DeclareMathOperator{\fold}{fold}
\DeclareMathOperator{\scan}{scan}
\author{James Geddes}
\date{\today}
\title{Grid types}
\hypersetup{
 pdfauthor={James Geddes},
 pdftitle={Grid types},
 pdfkeywords={},
 pdfsubject={},
 pdfcreator={Emacs 27.2}, 
 pdflang={English}}
\begin{document}

\maketitle

\section{Introduction}

Our language Cell exports a `generalised spreadsheet,' which is then processed
by Grid to produce an actual spreadsheet. This note is an attempt to describe a
possible definition for a `generalised spreadsheet.'

The language Remora
\cite{slepak:array:2014,remora:intro:2019,slepak:semantics:2019} resurrects the
idea of array-based programming that started with APL and J. Arrays look ideal
as a kind of grid-like data structure with which to generalise the
two-dimensional spreadsheet grid. However, to be usable as a spreadsheet, the
structure used by Grid must support expressions, which typically don't have the
form of a Remora-type array.

Recently Gibbons \cite{gibbons:aplicative:2017} generalised Remora's arrays to
what he calls `applicative, Naperian functors.' These are essentially container
types which support traversals, for which there is an `index' type such that
values of the container type are representable as maps from the index type to
the thing contained. For example, vectors of int of length 5 are equivalent to
maps from the set \((0, 1, 2, 3, 4)\) to the ints. The nice thing about these
types is that they seem to suport all the nice things one would want of arrays
(folds and so on) and are `lineariseable.'

It looks like applicative, Naperian functors are sufficiently general to
suppport expressions. However, they come with an awful lot of PL
machinery. Instead, in this note, we propose to start from the index types, and
build up arrays from there. The hope is that, first, all the properties we want
will be captured by some simple properties of the index types; and, second, that
this will provide a sufficiently rich set of array-like structures to capture
everything we need for Grid.


\section{Grids, vectors, and arrays}

A \emph{grid} is a totally ordered, finite set. If \(\GG\) is a grid, we denote
the ordering on \(\GG\) by \(<\) (at least when there is no confusion as to
which grid the order refers). Two grids, \(\gr{G}\) and \(\gr{H}\), are
\emph{isomorphic}, written \(\gr{G}\sim \gr{H}\), if there is an order-preserving
bijection between them. It is clear that all grids are isomorphic to one of the
bounded naturals, \((0, 1, \dotsc, N-1)\) for some \(N\). (Proof?) However, in
Cell we distinguish grids that are isomorphic but not identical.

We write \(\void\) for the empty grid (which is unique) and $\unit$ for a
distinguished grid with a single element. 

Fix, once and for all, a set, \(P\), of \emph{primitives}. (We will later choose
the primitives to be the atomic types of spreadsheets together with the
first-order functions. The primitives are the types of the values that go in
cells in a spreadsheet.) An \emph{array} is a map $\gr{G} \to P$, where $\gr{G}$
is any grid. Suppose $X$ is an array: we call a map $\gr{G} \to X$ a \emph{vector
of $X$}. (A vector of $X$ is also an array, of course.)

\emph{Example:} Given some primitive, say `42', the map $\unit \to P$ that assigns 42
to the single element of $\unit$ is an array. We write this array as simply the
value: \verb|42|.

\subsection{Direct sums of grids}

Let \((\GG_1, \GG_2, \dots, \GG_N)\) be a tuple of grids. The \emph{direct sum},
\(\bigoplus_i \GG_i\), is a grid such that
\begin{enumerate}
\item The set \(\bigoplus_i \GG_i\) is the disjoint union of the \(\GG_i\);
\item \(g<h\) in \(\bigoplus \GG_i\) if either (i) \(g, h\in \GG_i\) for some \(i\) and \(g<h\)
in \(\GG_i\) or (ii) \(g\in\GG_i\) and \(h\in\GG_j\) and \(i < j\).
\end{enumerate}
That is, the direct sum is what you get if you `line up the component grids in
order.' We write the direct sum of two grids \(\gr{G}\) and \(\gr{H}\) as
\(\gr{G}\oplus \gr{H}\). The direct sum of no grids is $\void$ and, for any grid
$\GG$, \(\GG\oplus\void = \GG\). For any \(X\), a map \(\bigoplus \GG_i\to X\) is
equivalent to an \(N\)-tuple of maps \((f_1, f_2, \dots, f_N)\), where \(f_i :
\GG_i\to X\).

The `standard grid of length $n$', denoted $\mathbf{n}$, is
\begin{equation*}
\mathbf{n} \equiv \bigoplus_{i=0}^{n-1} \unit.
\end{equation*}

\emph{Example}: Let $\one = \bigoplus\unit $, $\two = \bigoplus_{i=0}^1 \unit$ and
$\three = \bigoplus_{i=0}^2 \unit$ be standard grids of length 1, 2, and 3
respectively. The array of type $\one \to P$ that assigns the value `42' to the
single element of $\one$ is written \verb|[42]|. The array of type $\two \to P$
that assigns the value `10' to the first element of $\two$ and `20' to the
second element is written \verb|[10 20]|. Similarly, \verb|[10 20 30]| is an
array of type $\three \to P$. 

The kinds of arrays found in Remora are a subclass of what we are calling arrays
here. A rank 1 Remora array is a map \(\mathbf{n}\to P\) for some standard grid,
\(\mathbf{n}\). A rank 2 Remora array is a map from a standard grid,
\(\mathbf{m}\), to a rank 1 Remora array; and so on. Thus the type of a Remora
array is always specified completely by the list of integers \(m, n, \dotsc\).

Remora arrays are clearly arrays but the converse is not necessarily
true. Consider the map \(\mathbf{3}\to P\) whose values on the elements of
\(\mathbf{3}\) are 10, 20, and 30, in that order. This is a rank 1 Remora array,
\verb|[10 20 30]|. Likewise, consider the array \verb|[40 50]|, which is a map
\(\mathbf{2}\to P\). What can we say about some map \(\mathbf{3}\oplus\mathbf{2}\to P\)?
By our definition, that is an array; it is equivalent to a pair of maps, one of
type \(\mathbf{3}\to P\) and one of type \(\mathbf{2}\to P\). We might write an
example of this type as \texttt{[[10 20 30] [40 50]]}.\footnote{We don't write
it like this, but we might.} But this is not a Remora array: a Remora array
can't contain arrays of different length.

It is true that \(\mathbf{3}\oplus\mathbf{2}\) is isomorphic to \(\mathbf{5}\).
Remora has an \emph{append} function, so that, for example \texttt{(append [10
    20 30] [40 50])} is valid Remora and evaluates to \texttt{[10 20 30 40 50]}.
But isomorphism (in the sense we defined it above) is not identity.

This is why the direct sum is defined for $n$ factors for each $n$. In our model
there is a difference between, say, $\mathbf{3} = \bigoplus_3 \unit$ and $\unit
\oplus (\unit \oplus \unit)$.



\subsection{Direct products of grids}

In Remora, arrays have a rank. A rank 1 array is a vector of values; a rank 2
array is a vector of rank 1 arrays of the same length; and so on.  For example
\verb|[[1 2] [10 20] [100 20]]| is a rank 2 array. It is a vector (of length
three) of rank 1 arrays (of length 2). In our language, it is a map from $\three$
to arrays of type $\two \to P$. So the overall type of this array is $\three \to
(\two \to P)$.

We often imagine that this array is actually based on a `rectangular' grid, of
size $3 \times 2 = 6$. Let $\gr{G}$ and $\gr{H}$ be grids. The \emph{direct product},
$\gr{G}\otimes\gr{H}$, is a grid such that
\begin{enumerate}
\item The set $\gr{G}\otimes\gr{H}$ is the Cartesian product $\gr{G} \times \gr{H}$;
 \emph{i.e.}, the set of pairs $(g, h)$ where $g\in\gr{G}$ and $h\in\gr{H}$;
\item $(g, h) < (g', h')$ iff either $g < g'$ or $g = g'$ and $h < h'$. 
\end{enumerate}

%% OLD: When we did tuple products
%%
%% Let \((\GG_1, \GG_2, \dots, \GG_N)\) be a tuple of grids as before. The
%% \emph{direct product}, \(\bigotimes_i \GG_i\), is a grid such that
%% \begin{enumerate}
%% \item The set \(\bigotimes_i \GG_i\) is the Cartesian product of the \(\GG_i\);
%%   \emph{i.e.}, the set of tuples \((g_1, g_2, \dotsc, g_N)\) where
%%   \(g_i\in\GG_i\);
%% \item \((g_1, g_2, \dotsc, g_N) < (h_1, h_2, \dotsc, h_N)\) iff there is some
%%   \(k\in\{1, \dotsc, N\}\) such that (i) \(g_i = h_i\) for all \(i<k\); and (ii)
%%   \(g_k<h_k\).
%% \end{enumerate}

That is, the direct product is the Cartesian product with `dictionary order' on
the elements. The empty product is $\unit$.\footnote{Or, rather, I think that we
choose the distinguished unit grid $\unit$ to be the empty product.}
Furthermore, $\GG\otimes\void = \void$ (although I'm a little confused about whether
that's an `$=$' or a `$\sim$').

For any \(X\), a map $\gr{G}\otimes\gr{H} \to X\) is equivalent to a map $\gr{G} \to
(\gr{H} \to X)$. Thus, in the example, an array of type $\three \to (\two \to P)$ can
also be termed an array of type $\three \otimes \two \to P$.  

In Cell, no distinction is made between equivalent \emph{arrays} of this kind
and the right-associative map is taken to be canonical. (Note: this equivalence
seems to underlie the rank polymorphism inherent in array languages, but I am
not quite sure how to say this clearly.) Because of the canonical asociativity,
only the binary product is defined: there aren't $n$-ary products as there were
in the direct sum case.

A grid in Cell is either \(\void\), \(\unit\), $\mathbf{n}\sim \bigoplus_n \unit$,
or a direct sum or a direct product.

%% a x (b + c)
%% well, a x (b + c) -> A
%% is a -> (b+c) -> A
%% is a -> (b->A, c->A)
%% is (a->b->A, a->c->A)
%% is (a x b -> A, a x c -> A)
%% is (a x b + a x c) -> A
%% (a + b) x c
%% well, (a + b) x c -> A
%% is (a+b) -> c -> A
%% is (a -> c -> A, b -> c -> A)
%% is (a x c -> A, b x c -> A)
%% is (a x c + b x c) -> A

\subsection{Notation}

In Remora, a rank 2 array is a vector of rank 1 arrays. For example, the array
\verb|[[10 20 30] [100 200 300]]| is a map from $\two$ to arrays of type $\three
\to P$; we might write its type as $\two \to \three \to P$ (where the maps associate
to the right).

But how should we write an array that is a map $\three\oplus\three \to P$? It has the
same number of elements (namely, six) and a similar structure (a `pair of length
three vectors of $P$') but it is a different array because the underlying grids
are different. We write it has follows: \verb|{[10 20 30] [100 200 300]}|. The
braces distinguish the ``direct sum of two parts'' from ``a map from $\two$ to
the same parts.''

In this way, we can also write the array \verb|{[10 20] [100 200 300]}|, which
has type $\two\oplus\three \to P$, a type that does not exist in Remora. In Remora, the
list of components between matching pairs of square brackets must all be of the
same type and the interpretation is that the array is a map from a
\emph{standard} grid (to some array).

There is one wrinkle. We could also have written, say, \verb|[1 2 3]| as
\verb|{1 2 3}|, but it is convenient to have the convention that an array of
values whose grids is a direct sum of $\unit$ is written with square brackets. 

In summary, given arrays $\alpha : \gr{A}\to P$, $\beta : \gr{B}\to P$, and $\gamma : \gr{C}\to P$
the notation \texttt{\{$\alpha$ $\beta$ $\gamma$\}} means an array of type $\bigoplus
\{\gr{A}, \gr{B}, \gr{C}\}$. In the particular case that $\gr{A} = \gr{B} =
\gr{C}$ the notation \texttt{[$\alpha$ $\beta$ $\gamma$]} means an array of type $\three\to
(\gr{A}\to P)$.



\subsection{Further examples of direct sums}

In this section we describe examples of direct sums of grids and arrays built on
them. We write an array of type $\unit\to P$ as a `bare value,' such as
\verb|42|.

An array of type $\three \to P$ is conceptually a tree (see Figure~\ref{fig:gridthree}).
\begin{figure}[ht]
  \includegraphics[scale=0.7]{diags/grids.gv.pdf}
  \caption{The array \texttt{[10 20 30]}.}\label{fig:gridthree}
\end{figure}
Notice that the sum has three children because the direct sum in this case is a sum of
three arguments. By contrast, the array \verb|{10 [20 30]}| is represented as
shown in Figure~\ref{fig:gridsum}.
\begin{figure}[ht]
  \includegraphics[scale=0.7]{diags/grids.gv.2.pdf}
  \caption{The array \texttt{\{10 [20 30]\}}.}\label{fig:gridsum}
\end{figure}
Finally, note that the array \verb|[10]| is represented as a node with a single
child, as in Figure~\ref{fig:gridsingleton}, and not as a bare value (which
would simply be \verb|10|).
\begin{figure}[ht]
  \includegraphics[scale=0.7]{diags/grids.gv.3.pdf}
  \caption{The array \texttt{[10]}.}\label{fig:gridsingleton}
\end{figure}


\section{Expressions}

The kinds of arrays that are passed to Grid are not simply arrays of values,
just as an Excel spreadsheet is not simply a table of numbers. Instead, a
spreadsheet is able to express certain simple computations. So we need some way
to represent expressions as arrays.

To begin with, we assume that the set of primitives, $P$, is the disjoint union,
$F \uplus V$, of a set, $F$, of \emph{primitive functions} and a set, $V$, of
\emph{values}. A \emph{value array} is an array whose codomain is~$V$. In other
words, a value array is `an array, all of whose elements are values.' Likewise,
a \emph{function array} is an array whose codomain is~$F$.

With each primitive function there is associated a grid, known as the
\emph{signature} of the function. A primitive function, $f_0$, having signature
$\gr{S}$, is a first-order function from arrays to values, $f_0: (\gr{S} \to V) \to
V$. It maps arrays equivalent to $\gr{S}\to V$ to values. (By convention,
primitive functions have a subscript~0.) For example, the signature of primitive
binary addition is $\mathbf{2}$:
\begin{equation*}
+_0: (\mathbf{2} \to V) \to V.
\end{equation*}
Note that $V$ is a rank zero array, which is the same as a map from the empty
product to $V$. But the empty product is $\unit$, so $V$ is equivalent to
$\unit\to V$. Hence we might equivalently write:
\begin{equation*}
+_0: (\mathbf{2} \to V) \to (\unit \to V).
\end{equation*}

For most primitive functions one might think of, the signature is in fact a
standard grid: usually $\one$, $\two$, or sometimes $\three$. That is, there are
no functions which operate on ranges and in particular no `array functions.'
Perhaps the only exceptions are \texttt{INDEX()} and \texttt{MATCH()}. For this
version, we assume that all signatures are standard grids.

An \emph{simple application array} is an array $A$ of type (strictly, isomorphic
to the type) $\unit \oplus \gr{G} \to P$, where $A$ decomposes as $(\unit\to F, \GG \to
V)$. That is, it is a pair of a primitive function and an array of values. The
interpretation here is that the values are the arguments to the function. 



For example, consider the case where $\alpha = P$. Then the array has type
$\gr{a}\oplus\gr{G} \to P$. The intent is that the $\gr{a}$ component should be a
primitive function (\emph{i.e.}, a function array with but one function) and the
$\gr{G}$ component should be a value array. If the array is of that form and, in
addition, $\gr{G}$ is the signature of the primitive function, then we say that
the array is \emph{well-formed}. The interpretation of a well-formed array is
that it represents the application of a primitive function to an argument.



Notes: In Excel, absolute vs. relative references are distinguished by whether
the frame/cell things is happening.





An \emph{expression array} is either a value array or a well-formed application
array.

Notes:

Unit grid is not the same as the standard grid of length 1.
Units (``dot'') get absorbed in products
Now have to figure out array extension in grid

                                                                                                   



%% An \emph{application array} is an array $\mathgr{A}\to\dotsm\to P$ such that
%% \begin{enumerate}
%% \item $\mathgr{A} = \unit \oplus \mathbf{n}$ for some~$\mathbf{n}$; and
%% \item 
%% \end{enumerate}




\section{OLD: Being re-thought.}
\label{sec:org0f361a3}


A \emph{vector} is a map \(v : \GG\to S\) for some grid \(\GG\). We say that \(v\) is of
type ‘vector over \(\GG\),’ written and write this type as \(S^\GG\). By such a map,
we mean an assignment of a scalar to every value of type \(\GG\). So, for example,
supposing \(\GG = (\unit\oplus\unit) \oplus \unit\), there are three values of
type \(\GG\), not two; even though this particular \(\GG\) is a binary
sum. Equivalently, a map \((\unit\oplus\unit) \oplus \unit\to S\) is a pair of
maps: one \(\unit\oplus\unit\to S\) and one \(\unit\to S\). The former is of course
itself a pair of maps.

The idea here is that sum-type grids are the index types for vectors. Normally,
one introduces only a binary sum and then sums of greater numbers of types are
made up of repeated binary sums. However, we want vectors like \texttt{[10 20 30]}, and
we don't want to think of this as either \texttt{[10 [20 30]]} or \texttt{[ [10 20] 30]}.  So
there's a sense in which the kind of vectors we want are ‘associative.’ On the
other hand, we would \emph{also} like ‘vectors’ like \texttt{[10 [20 30]]}. An ‘associative
vector’ is a map, say, \(\bigoplus(\unit, \unit, \unit)\to S\), whereas the
previous example is a map \(\unit\oplus (\unit\oplus \unit)\to S\).

\subsubsection{Lifting functions to arrays}
\label{sec:orgc0391ab}

Let \(\GG\) be a grid and \(\alpha:\GG\to S\) a vector. Suppose \(f:S\to S\) is some
unary function on scalars. Then we can ‘lift’ \(f\) to a function on vectors by
function composition: \(f(\alpha) = f\circ\alpha\).

Suppose \(\alpha\) and \(\beta\) are vectors \(\GG\to S\). We obtain a function
\(\alpha\otimes\beta : \GG\to S\times S\). Namely, for each \(\mathfrak{g}\in \GG\),
form the pair \((\alpha(\mathfrak{g}), \beta(\mathfrak{g}))\). Thus, given a
binary function \(h : S\times S\to S\), we can likewise lift this function to
vectors by function composition: \(h(\alpha, \beta) = h\circ
(\alpha\otimes\beta)\). This function is known as \emph{zip-with}.

Suppose \(\alpha : \gr{G}\to S\) and \(\beta : \gr{H}\to S\) are vectors. We obtain
a vector \((\alpha\oplus\beta) : \gr{G}\oplus\gr{H}\to S\). Namely, an element of
\(\gr{G}\oplus\gr{H}\) is either an element of \(\gr{G}\) or an element of
\(\gr{H}\). If the former, apply \(\alpha\); if the latter, apply \(\beta\). 

An \emph{array} is a vector of the form \(\gr{G}\otimes\gr{H}\to S\). (TODO: Maybe it's
a vector of the form \(\bigotimes_i \gr{G}_i\to S\)?) Note that by definition of
\(\otimes\), the array \(\alpha\) is equivalent to a map with domain \(\gr{G}\) and
range \(S^\gr{H}\). That is, an array is ‘a sequence of vectors all of the same
shape.’

Suppose \(f:S^\gr{H}\to X\) is function that takes vectors over \(\gr{H}\) to
some \(X\) (which we leave open for now). Given an array \(\alpha : \gr{G}\otimes
\gr{H}\to S\), that is, a map \(\alpha : \gr{G}\to S^\gr{H}\), we lift \(f\) to
such arrays, again by function composition.

What kind of thing could \(X\) be? One obvious possibility is \(S\): for example, if
the function is ‘sum.’ But it could also be some other vector: for example, if
the function computes the prefix sums. 

\subsubsection{Reductions and folds}
\label{sec:org1f2a8ac}

A  \emph{first-order  function}  is  one  whose  domain  and  range  are  scalars  or
vectors. (TODO: Probably an element of \(S\)  counts as a vector?) The above shows
how, given first-order functions on  scalars, one can make first-order functions
on vectors which don't change the underlying grid. (Note that mapping produces a
vector of the same `shape' as the input.)

Every grid is either a sum or a product. So the question reduces to the
operations that one might imagine on sums or products. 

Suppose \(\gr{G} = \bigoplus_{i = 1}^N \gr{G}_i\) is a sum. If \(f\) is an \(N\)-ary
function, whose \(i\)th argument is a vector over \(\gr{G}_i\), then we can apply
\(f\) immediately to a vector over \(\gr{G}\).

Consider the particular case where \(f\) is a binary, associative operator -- that
is \(f(f(x, y), z) = f(x, f(y, z))\) and both arguments of \(f\) are of the same
type. In this case we can apply \(f\) to any \emph{array} whose ‘elements have the
appropriate type.’ That operation is called \emph{reduce}.

On the other hand, if \(f\) is a binary operator, not necessarily associative,
then we can apply it recursively to a ‘list of pairs’ type. This operation is
\emph{fold}.

(Note: Are we asserting that \emph{all} operations on sum types must be associative?)

TODO: \emph{scan} and \emph{trace}. 

TODO: \emph{replicate}

\subsubsection{Transposition}
\label{sec:org68e418b}

\(\gr{G}\otimes\gr{H}\) is clearly not the same grid as
\(\gr{H}\otimes\gr{G}\). However, there is clearly a correspondence between a
vector \(\gr{G}\otimes\gr{H}\to S\) and a vector \(\gr{H}\otimes\gr{G}\to
S\); namely, given a pair \((\gr{h}, \gr{g})\), reverse the order and apply the
given vector. This operation is called \emph{transposition}. 

\subsection{The nature of polymorphism}
\label{sec:orgd61f222}

The polymorphism so far -- ‘rank polymorphism’ -- is all that which arises from
function composition. If I have a function \(f:B\to X\), then, for any function
\(A\to B\) I get a function \(A\to X\). It's sort of ‘polymorphic in \(A\)’:

$$
\forall A, (f\circ) : (A\to B)\to (A\to X)
$$

Reduction is similar but changes the structure. It applies to a map from any
sum-type to a particular type and returns a value of that particular type. Eg,
\texttt{(reduce +)} applies to any \(A\to\R\), producing an \(\R\), because
\(+:\R\times\R\to\R\):

$$
\forall A, (\operatorname{reduce}\, +) : (A \to \R)\to \R.
$$

However, there is some ambiguity. 

Suppose \(\gr{A}\) and \(\gr{B}\) are grids, and \(\alpha : \gr{A}\otimes\gr{B}\to\R\)
is an array. The domain of \(\alpha\) is not a sum type, so we can't reduce
directly. Here are three ways we might interpret the application of reduction. 

\begin{enumerate}
\item We might interpret \(\gr{A}\otimes\gr{B}\) as 

$$
   \gr{A}\otimes\gr{B} \sim \bigoplus_{\gr{A}, \gr{B}} \unit  
   $$

and then apply reduce. In other words “sum over all elements of the
two-dimensional array.” My sense is that this is not the typical
interpretation; perhaps because the same outcome can be achieved by doing the
next two things in sucession.

\item We might interpret the type of \(\alpha\) as

$$
   \gr{A}\otimes\gr{B}\to\R \sim \gr{A} \to (\gr{B}\to \R),
   $$

instantiate \((\reduce +)\) at \(\gr{B}\to\R\), and apply function
composition. In other words, we reduce “each of the inner vectors of
\(\alpha\),” leaving a vector over the same grid as \(\alpha\).

I think this is the default for reduction, corresponding to “find the cell
and map over the frame” approach of Remora.

\item Finally, we might “lift” \(+\) from \(\R\times \R\) to \(X\otimes X\), where \(X =
   \gr{B}\to\R\) by using \(\zipWith\). Now we can instantiate the reduction at
\(\gr{A}\to X\)
\end{enumerate}

The last approach is often explained by using transposition, but there was no
transposition here. Is there an identity, perhaps? Something along the lines of:

$$
(\reduce +) \circ \operatorname{transpose} = (\reduce (\zipWith +)) ? 
$$

\section{Using arrays to represent first-order expressions}
\label{sec:orgac340af}

A \emph{first-order array function} is a grid, \(\gr{G}\), and a map \(f: (\gr{G}\to
S)\to S\). That is, it's a map from arrays over a certain grid to scalars. 

For example, \(\operatorname{add}_2 : (\mathbf{2}\to S)\to S\) is a first-order
array function which acts on vectors of length 2 (by adding up their values):
$$
\operatorname{add}_2([10\; 20]) = 30.
$$ 
Notice that the value is a scalar, not an array of length 1.

We assume here that first-order array functions are simply given as set \(B\) of
built-ins.  (In reality, one constructs them from functions on scalars applied
to values at particular grid indices.) We enlarge \(S\) (the scalars) by the set
of built-ins: \(S^* = S\cup B\).

TODO: \((\reduce +)\) at \(\mathbf{4}\) (say) is also a first-order array function.  

We assume the existence of a grid \(\gr{F}\sim\unit\). An \emph{array expression} is
either:
\begin{itemize}
\item an array of \(S\); or
\item an array of type \(\gr{F}\oplus\gr{G}\to S^*\); where the first element
\(\gr{F}\to B\) is a first-order array function of type \(\gr{G}\to S\); or
\item an array of type \(\gr{F}\oplus\gr{G}\to X\) where \(X\) is an array expression.
\end{itemize}

Array expressions are \emph{evaluated} to produce arrays. The value of an array of
\(S\) is just that array. The value of an expression whose first element is a
first-order function is obtained by evaluating the remainder and then applying
that first-order function.

\section{Possible Racket-y syntax version}
\label{sec:orga1852e4}


\section{References}
\label{sec:orgbf56ab8}

\label{org334ac2d}
\bibliographystyle{unsrt}

\label{orgefbceba}
\bibliography{nocell}
\end{document}
