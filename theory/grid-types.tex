\documentclass[11pt]{article}
%\usepackage{graphicx}
%\usepackage{grffile}
%\usepackage{longtable}
%\usepackage{wrapfig}
%\usepackage{rotating}
%\usepackage[normalem]{ulem}
\usepackage{amsmath}
\usepackage{textcomp}
\usepackage{amssymb}
%\usepackage{capt-of}
\usepackage{hyperref}
%\usepackage{fontspec} -- was added by org-mode
%\usepackage{newfloat}
%\usepackage{minted} -- was added by org-mode
\usepackage{booktabs}
\newcommand{\gr}[1]{\mathfrak{#1}}
\newcommand{\GG}{\gr{G}}
\newcommand{\VV}{\mathbb{V}}
\newcommand{\R}{\mathbb{R}}
\newcommand{\void}{\mathbf{0}}
\newcommand{\unit}{\mathbf{1}}
\DeclareMathOperator{\reduce}{reduce}
\DeclareMathOperator{\map}{map}
\DeclareMathOperator{\zipWith}{zipWith}
\DeclareMathOperator{\fold}{fold}
\DeclareMathOperator{\scan}{scan}
\author{James Geddes}
\date{\today}
\title{Grid types}
\hypersetup{
 pdfauthor={James Geddes},
 pdftitle={Grid types},
 pdfkeywords={},
 pdfsubject={},
 pdfcreator={Emacs 27.2}, 
 pdflang={English}}
\begin{document}

\maketitle



\section{Introduction}
\label{sec:org37b3f04}

Our language Cell exports a `generalised spreadsheet,' which is then processed
by Grid to produce an actual spreadsheet. This note is an attempt to describe a
possible definition for a `generalised spreadsheet.'

The language Remora
\cite{slepak:array:2014,remora:intro:2019,slepak:semantics:2019} resurrects the
idea of array-based programming that started with APL and J. Arrays look ideal
as a kind of grid-like data structure with which to generalise the
two-dimensional spreadsheet grid. However, to be usable as a spreadsheet, the
structure used by Grid must support expressions, which typically don't have the
form of a Remora-type array.

Recently Gibbons \cite{gibbons:aplicative:2017} generalised Remora's arrays to
what he calls `applicative, Naperian functors.' These are essentially container
types which support traversals, for which there is an `index' type such that
values of the container type are representable as maps from the index type to
the thing contained. For example, vectors of int of length 5 are equivalent to
maps from the set \((0, 1, 2, 3, 4)\) to the ints. The nice thing about these
types is that they seem to suport all the nice things one would want of arrays
(folds and so on) and are `lineariseable.'

It looks like applicative, Naperian functors are sufficiently general to
suppport expressions. However, they come with an awful lot of PL theory
machinery. Instead, in this note, we propose to start from the index types, and
build up arrays from there. The hope is that, first, all the properties we want
will be captured by some simple properties of the index types; and, second, that
this will provide a sufficiently rich set of array-like structures to capture
everything we need for Grid.


\section{Grids and arrays}
\label{sec:org26c669c}

A \emph{grid} is a totally ordered, finite set. If \(\GG\) is a grid, we denote the
ordering on \(\GG\) by \(<\) where there is no confusion as to which grid the order
refers. Two grids, \(\gr{G}\) and \(\gr{H}\), are \emph{isomorphic}, written \(\gr{G}\sim
\gr{H}\), if there is an order-preserving bijection between them. It is clear that
all grids are isomorphic to one of the bounded naturals, \((0, 1, \dotsc, N-1)\)
for some \(N\). (Proof?)

We write \(\void\) for the empty grid; \(\mathbf{1}\) for the grid \((0)\); and in
general \(\mathbf{n}\) for the grid \((0, 1, \dotsc, n-1)\). 

Fix, once and for all, a set \(S\) of \emph{primitives}. An \emph{array of rank 0} is an
element of \(S\). An \emph{array of rank \(n\)} is a map from a grid to an array of rank
\(n-1\). (Thus an array of rank 1 is a map from a grid to \(S\).)

Let \((\GG_1, \GG_2, \dots, \GG_N)\) be a tuple of grids. The \emph{direct sum},
\(\bigoplus_i \GG_i\), is a grid such that
\begin{enumerate}
\item The set \(\bigoplus_i \GG_i\) is the disjoint union of the \(\GG_i\);
\item \(g<h\) in \(\bigoplus \GG_i\) if either (i) \(g, h\in \GG_i\) for some \(i\) and \(g<h\)
in \(\GG_i\) or (ii) \(g\in\GG_i\) and \(h\in\GG_j\) and \(i < j\).
\end{enumerate}
That is, the direct sum is what you get if you `line up the component grids in
order.' We write the direct sum of two grids \(\gr{G}\) and \(\gr{H}\) as \(\gr{G}\oplus
\gr{H}\). For any grid, \(\GG\oplus\void \sim \GG\). Furthermore,
\(\bigoplus_{i=1}^n \mathbf{1} \sim \mathbf{n}\). 

For any \(X\), a map \(\bigoplus \GG_i\to X\) is an \(N\)-tuple of maps \((f_1, f_2,
\dots, f_N)\), where \(f_i : \GG_i\to X\).

Let \((\GG_1, \GG_2, \dots, \GG_N)\) be a tuple of grids as before. The \emph{direct
product}, \(\bigotimes_i \GG_i\), is a grid such that
\begin{enumerate}
\item The set \(\bigotimes_i \GG_i\) is the Cartesian product of the \(\GG_i\); \emph{i.e.},
the set of tuples \((g_1, g_2, \dotsc, g_N)\) where \(g_i\in\GG_i\);
\item \((g_1, g_2, \dotsc, g_N) < (h_1, h_2, \dotsc, h_N)\) iff there is some
\(k\in\{1, \dotsc, N\}\) such that (i) \(g_i = h_i\) for all \(i<k\); and (ii) \(g_k<h_k\).
\end{enumerate}
That is, the direct product is the Cartesian product with `dictionary order' on
the elements. We write the direct product of two grids, \(\gr{G}\) and \(\gr{H}\) as
\(\gr{G}\otimes\gr{H}\). The empty product is \(\unit\); \(\GG\otimes\unit\sim \GG\);
and \(\GG\otimes\void \sim \void\).

For any \(X\), a map \(\bigotimes_i\GG_i \to X\) is a map \(\GG_1\to (\GG_2\to
(\dotsm (\GG_N \to X))\dotsm )\).

A grid in Cell is either \(\void\), \(\unit\), or a direct sum or a direct product. 

The kinds of arrays found in Remora are a subclass of what we are calling arrays
here. A rank 1 Remora array is specifically a map \(\mathbf{n}\to S\) for some
grid \(\mathbf{n}\). Rank 2 Remora arrays are maps from \(\mathbf{m}\) to a rank 1
Remora array; and so on. Thus the type of a Remora array is always specified
completely by the list of integers \(m, n, \dotsc\).

Remora arrays are clearly arrays but the converse is not necessarily
true. Consider the map \(\mathbf{3}\to S\) whose values on the elements of
\(\mathbf{3}\) are 10, 20, and 30, in that order. This is a rank 1 Remora array,
written \texttt{[10 20 30]}. Likewise, consider the array \texttt{[40 50]}, which is a map
\(\mathbf{2}\to S\). What can we say about some map \(\mathbf{3}\oplus\mathbf{2}\to
S\)? By our definitions, that is an array; it is equivalent to a pair of maps,
one of type \(\mathbf{3}\to S\) and one of type \(\mathbf{2}\to S\). We might
reasonably write an example of this type as \texttt{[[10 20 30] [40 50]]}. But this is
not a Remora array: a Remora array can't contain arrays of different length.

It is true that \(\mathbf{3}\oplus\mathbf{2}\) is isomorphic to \(\mathbf{5}\).
Remora has an \emph{append} function, so that, for example \texttt{(append [10 20 30] [40
50])} is valid Remora and evaluates to \texttt{[10 20 30 40 50]}. But isomorphism (in
the sense we defined it above) is not identity.

This is why it 

a x (b + c)
well, a x (b + c) -> A
is a -> (b+c) -> A
is a -> (b->A, c->A)
is (a->b->A, a->c->A)
is (a x b -> A, a x c -> A)
is (a x b + a x c) -> A
(a + b) x c
well, (a + b) x c -> A
is (a+b) -> c -> A
is (a -> c -> A, b -> c -> A)
is (a x c -> A, b x c -> A)
is (a x c + b x c) -> A




\section{OLD: Being re-thought.}
\label{sec:org0f361a3}


A \emph{vector} is a map \(v : \GG\to S\) for some grid \(\GG\). We say that \(v\) is of
type ‘vector over \(\GG\),’ written and write this type as \(S^\GG\). By such a map,
we mean an assignment of a scalar to every value of type \(\GG\). So, for example,
supposing \(\GG = (\unit\oplus\unit) \oplus \unit\), there are three values of
type \(\GG\), not two; even though this particular \(\GG\) is a binary
sum. Equivalently, a map \((\unit\oplus\unit) \oplus \unit\to S\) is a pair of
maps: one \(\unit\oplus\unit\to S\) and one \(\unit\to S\). The former is of course
itself a pair of maps.

The idea here is that sum-type grids are the index types for vectors. Normally,
one introduces only a binary sum and then sums of greater numbers of types are
made up of repeated binary sums. However, we want vectors like \texttt{[10 20 30]}, and
we don't want to think of this as either \texttt{[10 [20 30]]} or \texttt{[ [10 20] 30]}.  So
there's a sense in which the kind of vectors we want are ‘associative.’ On the
other hand, we would \emph{also} like ‘vectors’ like \texttt{[10 [20 30]]}. An ‘associative
vector’ is a map, say, \(\bigoplus(\unit, \unit, \unit)\to S\), whereas the
previous example is a map \(\unit\oplus (\unit\oplus \unit)\to S\).

\subsubsection{Lifting functions to arrays}
\label{sec:orgc0391ab}

Let \(\GG\) be a grid and \(\alpha:\GG\to S\) a vector. Suppose \(f:S\to S\) is some
unary function on scalars. Then we can ‘lift’ \(f\) to a function on vectors by
function composition: \(f(\alpha) = f\circ\alpha\).

Suppose \(\alpha\) and \(\beta\) are vectors \(\GG\to S\). We obtain a function
\(\alpha\otimes\beta : \GG\to S\times S\). Namely, for each \(\mathfrak{g}\in \GG\),
form the pair \((\alpha(\mathfrak{g}), \beta(\mathfrak{g}))\). Thus, given a
binary function \(h : S\times S\to S\), we can likewise lift this function to
vectors by function composition: \(h(\alpha, \beta) = h\circ
(\alpha\otimes\beta)\). This function is known as \emph{zip-with}.

Suppose \(\alpha : \gr{G}\to S\) and \(\beta : \gr{H}\to S\) are vectors. We obtain
a vector \((\alpha\oplus\beta) : \gr{G}\oplus\gr{H}\to S\). Namely, an element of
\(\gr{G}\oplus\gr{H}\) is either an element of \(\gr{G}\) or an element of
\(\gr{H}\). If the former, apply \(\alpha\); if the latter, apply \(\beta\). 

An \emph{array} is a vector of the form \(\gr{G}\otimes\gr{H}\to S\). (TODO: Maybe it's
a vector of the form \(\bigotimes_i \gr{G}_i\to S\)?) Note that by definition of
\(\otimes\), the array \(\alpha\) is equivalent to a map with domain \(\gr{G}\) and
range \(S^\gr{H}\). That is, an array is ‘a sequence of vectors all of the same
shape.’

Suppose \(f:S^\gr{H}\to X\) is function that takes vectors over \(\gr{H}\) to
some \(X\) (which we leave open for now). Given an array \(\alpha : \gr{G}\otimes
\gr{H}\to S\), that is, a map \(\alpha : \gr{G}\to S^\gr{H}\), we lift \(f\) to
such arrays, again by function composition.

What kind of thing could \(X\) be? One obvious possibility is \(S\): for example, if
the function is ‘sum.’ But it could also be some other vector: for example, if
the function computes the prefix sums. 

\subsubsection{Reductions and folds}
\label{sec:org1f2a8ac}

A  \emph{first-order  function}  is  one  whose  domain  and  range  are  scalars  or
vectors. (TODO: Probably an element of \(S\)  counts as a vector?) The above shows
how, given first-order functions on  scalars, one can make first-order functions
on vectors which don't change the underlying grid. (Note that mapping produces a
vector of the same `shape' as the input.)

Every grid is either a sum or a product. So the question reduces to the
operations that one might imagine on sums or products. 

Suppose \(\gr{G} = \bigoplus_{i = 1}^N \gr{G}_i\) is a sum. If \(f\) is an \(N\)-ary
function, whose \(i\)th argument is a vector over \(\gr{G}_i\), then we can apply
\(f\) immediately to a vector over \(\gr{G}\).

Consider the particular case where \(f\) is a binary, associative operator -- that
is \(f(f(x, y), z) = f(x, f(y, z))\) and both arguments of \(f\) are of the same
type. In this case we can apply \(f\) to any \emph{array} whose ‘elements have the
appropriate type.’ That operation is called \emph{reduce}.

On the other hand, if \(f\) is a binary operator, not necessarily associative,
then we can apply it recursively to a ‘list of pairs’ type. This operation is
\emph{fold}.

(Note: Are we asserting that \emph{all} operations on sum types must be associative?)

TODO: \emph{scan} and \emph{trace}. 

TODO: \emph{replicate}

\subsubsection{Transposition}
\label{sec:org68e418b}

\(\gr{G}\otimes\gr{H}\) is clearly not the same grid as
\(\gr{H}\otimes\gr{G}\). However, there is clearly a correspondence between a
vector \(\gr{G}\otimes\gr{H}\to S\) and a vector \(\gr{H}\otimes\gr{G}\to
S\); namely, given a pair \((\gr{h}, \gr{g})\), reverse the order and apply the
given vector. This operation is called \emph{transposition}. 

\subsection{The nature of polymorphism}
\label{sec:orgd61f222}

The polymorphism so far -- ‘rank polymorphism’ -- is all that which arises from
function composition. If I have a function \(f:B\to X\), then, for any function
\(A\to B\) I get a function \(A\to X\). It's sort of ‘polymorphic in \(A\)’:

$$
\forall A, (f\circ) : (A\to B)\to (A\to X)
$$

Reduction is similar but changes the structure. It applies to a map from any
sum-type to a particular type and returns a value of that particular type. Eg,
\texttt{(reduce +)} applies to any \(A\to\R\), producing an \(\R\), because
\(+:\R\times\R\to\R\):

$$
\forall A, (\operatorname{reduce}\, +) : (A \to \R)\to \R.
$$

However, there is some ambiguity. 

Suppose \(\gr{A}\) and \(\gr{B}\) are grids, and \(\alpha : \gr{A}\otimes\gr{B}\to\R\)
is an array. The domain of \(\alpha\) is not a sum type, so we can't reduce
directly. Here are three ways we might interpret the application of reduction. 

\begin{enumerate}
\item We might interpret \(\gr{A}\otimes\gr{B}\) as 

$$
   \gr{A}\otimes\gr{B} \sim \bigoplus_{\gr{A}, \gr{B}} \unit  
   $$

and then apply reduce. In other words “sum over all elements of the
two-dimensional array.” My sense is that this is not the typical
interpretation; perhaps because the same outcome can be achieved by doing the
next two things in sucession.

\item We might interpret the type of \(\alpha\) as

$$
   \gr{A}\otimes\gr{B}\to\R \sim \gr{A} \to (\gr{B}\to \R),
   $$

instantiate \((\reduce +)\) at \(\gr{B}\to\R\), and apply function
composition. In other words, we reduce “each of the inner vectors of
\(\alpha\),” leaving a vector over the same grid as \(\alpha\).

I think this is the default for reduction, corresponding to “find the cell
and map over the frame” approach of Remora.

\item Finally, we might “lift” \(+\) from \(\R\times \R\) to \(X\otimes X\), where \(X =
   \gr{B}\to\R\) by using \(\zipWith\). Now we can instantiate the reduction at
\(\gr{A}\to X\)
\end{enumerate}

The last approach is often explained by using transposition, but there was no
transposition here. Is there an identity, perhaps? Something along the lines of:

$$
(\reduce +) \circ \operatorname{transpose} = (\reduce (\zipWith +)) ? 
$$

\section{Using arrays to represent first-order expressions}
\label{sec:orgac340af}

A \emph{first-order array function} is a grid, \(\gr{G}\), and a map \(f: (\gr{G}\to
S)\to S\). That is, it's a map from arrays over a certain grid to scalars. 

For example, \(\operatorname{add}_2 : (\mathbf{2}\to S)\to S\) is a first-order
array function which acts on vectors of length 2 (by adding up their values):
$$
\operatorname{add}_2([10\; 20]) = 30.
$$ 
Notice that the value is a scalar, not an array of length 1.

We assume here that first-order array functions are simply given as set \(B\) of
built-ins.  (In reality, one constructs them from functions on scalars applied
to values at particular grid indices.) We enlarge \(S\) (the scalars) by the set
of built-ins: \(S^* = S\cup B\).

TODO: \((\reduce +)\) at \(\mathbf{4}\) (say) is also a first-order array function.  

We assume the existence of a grid \(\gr{F}\sim\unit\). An \emph{array expression} is
either:
\begin{itemize}
\item an array of \(S\); or
\item an array of type \(\gr{F}\oplus\gr{G}\to S^*\); where the first element
\(\gr{F}\to B\) is a first-order array function of type \(\gr{G}\to S\); or
\item an array of type \(\gr{F}\oplus\gr{G}\to X\) where \(X\) is an array expression.
\end{itemize}

Array expressions are \emph{evaluated} to produce arrays. The value of an array of
\(S\) is just that array. The value of an expression whose first element is a
first-order function is obtained by evaluating the remainder and then applying
that first-order function.

\section{Possible Racket-y syntax version}
\label{sec:orga1852e4}


\section{References}
\label{sec:orgbf56ab8}

\label{org334ac2d}
\bibliographystyle{unsrt}

\label{orgefbceba}
\bibliography{nocell}
\end{document}
